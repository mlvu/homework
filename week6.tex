\documentclass[11pt]{article}

\usepackage{color}
\usepackage[usenames,dvipsnames,svgnames,table]{xcolor}

\usepackage{graphicx}
\usepackage{amsmath}
\usepackage{amssymb}
\usepackage{eulervm}
\usepackage{charter}
\usepackage{exercise}
\PassOptionsToPackage{hyphens}{url}\usepackage{hyperref}
\hypersetup{colorlinks=true,linkcolor=DarkRed,urlcolor=DarkRed}
\usepackage{float}
\usepackage{enumitem}
\usepackage{bm}

\newcommand{\p}{\;\text{.}}

\newcommand{\R}{{\mathbb R}}
\newcommand{\N}{{\mathbb N}}
\newcommand{\E}{{\mathbb E}}

\newcommand{\U}{{\bm U}}
\newcommand{\V}{{\bm V}}
\newcommand{\W}{{\bm W}}
\newcommand{\A}{{\bm A}}
\newcommand{\B}{{\bm B}}
\newcommand{\C}{{\bm C}}
\newcommand{\ba}{{\bm a}}
\newcommand{\bb}{{\bm b}}
\newcommand{\boc}{{\bm c}}
\newcommand{\bu}{{\bm u}}
\newcommand{\bv}{{\bm v}}
\newcommand{\bw}{{\bm w}}
\newcommand{\x}{{\bm x}}
\newcommand{\y}{{\bm y}}
\newcommand{\z}{{\bm z}}
\newcommand{\X}{{\bm X}}
\newcommand{\Y}{{\bm Y}}
\newcommand{\Z}{{\bm Z}}
\newcommand{\bt}{{\bm t}}
\newcommand{\ma}[1]{\begin{pmatrix}#1\end{pmatrix}}

\DeclareMathOperator*{\argmin}{arg\,min}
\DeclareMathOperator*{\argmax}{arg\,max}

\newcommand{\ans}[2]{{\leavevmode\color{OliveGreen}#2}} % no answers
% \newcommand{\ans}[2]{{\leavevmode\color{OliveGreen}#1}} % answers

\newcommand{\red}[1]{{\textcolor{Red}{#1}}}
\definecolor{myred}{HTML}{C82506}
\definecolor{myblue}{HTML}{0365C0}
\definecolor{mygreen}{HTML}{00882B}
\definecolor{myorange}{HTML}{DE6A10}
\definecolor{mypurple}{HTML}{773F9B}
\definecolor{myyellow}{HTML}{DCBD23}

\newcommand{\rc}[1]{{\textcolor{myred}{ #1 }}}
\newcommand{\gc}[1]{{\textcolor{mygreen}{ #1 }}}
\newcommand{\bc}[1]{{\textcolor{myblue}{ #1 }}}
\newcommand{\kc}[1]{{\textcolor{Gray}{ #1 }}}
\newcommand{\oc}[1]{{\textcolor{myorange}{ #1 }}}
\newcommand{\pc}[1]{{\textcolor{mypurple}{ #1 }}}
\newcommand{\yc}[1]{{\textcolor{myyellow}{ #1 }}}

\newcommand{\kp}{{\kc{\partial}}}
% Exam questions
\newcommand{\trueorfalse}{\linebreak \textbf{A } True ~~ \textbf{B } False}
\newcommand{\choices}[4]{\linebreak\noindent\textbf{A } #1~~\textbf{B } #2~~\textbf{C } #3~~\textbf{D } #4}
%\newcommand{\choicesnum}[4]{\linebreak\noindent{\fill}\textbf{A } \np{#1}~~\textbf{B } \np{#2}~~\textbf{C } \np{#3}~~\textbf{D } \np{#4}}
%\newcommand{\twochoices}[2]{\linebreak\noindent\textbf{A } #1~~\textbf{B } #2}
%\newcommand{\threechoices}[3]{\linebreak\noinden\textbf{A } #1~~\textbf{B } #2~~\textbf{C } #3}
\newcommand{\choicesv}[4]{\\ \begin{minipage}{\textwidth}\vspace{4pt}\noindent\textbf{A } #1\\ \noindent\textbf{B } #2\\ \noindent\textbf{C } #3\\ \noindent\textbf{D } #4\end{minipage}}

% Homework questions
\newcommand{\qu}[0]{%
  \addtocounter{Question}{1}%
  \noindent \textbf{question \theQuestion:}
}


\newcommand{\an}{\ans{\checkmark}{}}

%\usepackage{tikz}
%\usepackage{tikz-qtree}
\usepackage{qtree}
\usepackage{xfrac}

\title{Week 6: Decision Trees and Ensembles}

\begin{document}

\maketitle

\section{Decision trees}

Imagine you do a newspaper round to help you get through these lean times.  On your round, you encounter a number of dogs that either bark or (try to) bite. The dogs are described by the following binary features: \emph{Heavy}, \emph{Smelly}, \emph{Big} and \emph{Growling}. Consider the following set of examples:

\begin{center}
\begin{tabular}{c  c  c  c  | c }
Heavy & Smelly &  Big & Growling & Bites \\
\hline
\rc{No}  & \rc{No} & \rc{No}  & \rc{No}  & \oc{No} \\
\rc{No}  & \rc{No} & \gc{Yes} & \rc{No}  & \oc{No} \\
\gc{Yes} & \gc{Yes}& \rc{No}  & \gc{Yes} & \oc{No} \\
\gc{Yes} & \rc{No} & \rc{No}  & \gc{Yes} & \bc{Yes} \\
\rc{No}  &\gc{Yes} & \gc{Yes} & \rc{No}  & \bc{Yes} \\
\rc{No}  & \rc{No} & \gc{Yes} & \gc{Yes} & \bc{Yes} \\
\rc{No}  & \rc{No} & \rc{No}  & \gc{Yes} & \bc{Yes} \\
\gc{Yes} &\gc{Yes} & \rc{No}  & \rc{No}  & \bc{Yes} \\
\hline
\end{tabular}
\end{center}


\qu What is the entropy of the target value \emph{Bites} in the data?
\ans{
\[H(\text{Bites})= - \sfrac{5}{8}\log_2\sfrac{5}{8} - \sfrac{3}{8}\log_2\sfrac{3}{8} \approx 0.9544\]
}{}


\qu Which attribute would the ID3 algorithm choose to use for the root of the tree (without pruning)?
\ans{Growling}{}


\qu What is the information gain of the attribute you chose in the previous question?
\ans{approximately 0.0487}{}


\qu Draw the full decision tree that would be learned for this data using ID3 without pruning.

\ans{

\begin{figure}[H]
\centering
\includegraphics[width=0.7\linewidth]{Tree}
\caption{Decision tree}
\end{figure}
}{}


\qu Suppose three new dogs appear in your round as listed in the table below. Classify them using the decision tree from the previous question.

\begin{center}
\begin{tabular}{c c c c c | c }
Dog & Heavy & Smelly &  Big & Growling & Bites \\
\hline
Buster & \gc{Yes} & \gc{Yes} & \gc{Yes} & \gc{Yes} & \ans{\oc{No}}{?} \\
Pluto  & \rc{No}  & \gc{Yes} & \rc{No}  & \gc{Yes} & \ans{\oc{No}}{?} \\
Zeus   & \gc{Yes} & \gc{Yes} & \rc{No}  & \rc{No} & \ans{\bc{Yes}}{?} \\
\hline
\end{tabular}
\end{center}

\qu Someone proposes a new scheme to prevent overfitting: she suggests to set a pre-defined maximum depth for the decision trees. When the standard algorithm reaches this depth, it terminates. Could this help to prevent overfitting? Why (not)?
\ans{
It ensures smaller, simpler trees (a smaller model space) and therefore can reduce the risk of overfitting. Overfitting is essentially memorizing too much of your data. Smaller trees can memorize less.
}{}

\qu In the maximum depth scheme introduced above, how would you determine a good value for the maximum depth for a given data set?

\ans{
The maximum depth is a \emph{hyperparameter}. We can choose good values for our hyperparameters by splitting our training data into a validation set and trying different values (either by cross validation or just single runs).}{}

\qu Why can't we apply L1-regularization to this decision tree learning problem?

\ans{L1 regularization assumes that your model is described by a real valued vector of parameters (i.e. your model space is continuous.) Decision trees have a \emph{discrete} model space.}{}

\section{Variational autoencoders}

The maximum likelihood principle tells us to optimize the quantity $p(x\mid \theta)$ as a function of $\theta$ (the model parameters).

For complex models, this does not usually lead to a closed form solution and applying gradient descent to $\theta$ directly tends to lead to mode collapse. Instead we rely on the following equality:

\[
\ln p(\x\mid \theta) = L(\rc{q}, \theta) + KL(\rc{q}, \gc{p}) \\
\]
with
\begin{align*}
\rc{q}(\z)&\; \text{any distribution on $\z$} \\
\gc{p}&= p(\z \mid \x, \theta) \\
KL(\rc{q}, \gc{p})&\; \text{the Kullback-Leibler divergence}\\
L(\rc{q}, \theta) &= \E_\rc{q} \ln \frac{p(\x, \z \mid \theta) }{\rc{q}(\z)}
\end{align*}

We will first prove that this equality holds. We start with the right hand side, fill in the components, and derive the left-hand side.

\qu Fill in the blanks. We have written everything in terms of \emph{expectations} $\E$ to simplify the notation. The expectation is over the random variable $\z$, while $\x$ has some definite value. Note that $\E f(\z) + \E g(\z) = \E\left [f(\z) + g(\z)\right]$.

\begin{align*}
L(\rc{q}, \theta) + KL(\rc{q}, \gc{p}) &= \ans{\E_\rc{q} \ln \frac{p(\x, \z \mid \theta) }{\rc{q}(\z)} -\E_\rc{q} \ln \frac{p(\z \mid \x, \theta)}{\rc{q}(\z)}}{\ldots}\\
&= \mathbb{E}_\rc{q} \ln \frac{p(\x, \z \mid \theta) }{\rc{q}(\z)} - \mathbb{E}_\rc{q} \ln \frac{p(\z \mid \x, \theta)}{\rc{q}(\z)}\\
&= \ans{\E_\rc{q}  \ln p(\x,\z\mid \theta) - \E_\rc{q} \ln \rc{q}(z) - \E_\rc{q} \ln p(\z\mid \x, \theta) + \E_\rc{q} \ln \rc{q}(z)}{\ldots}\\
&= \mathbb{E}_\rc{q} \ln p(\x, \z \mid \theta)  - \mathbb{E}_\rc{q} \ln p(\z \mid \x, \theta) \\
&= \mathbb{E}_\rc{q} \ln \frac{p(\x, \z \mid \theta) }{ p(\z \mid \x, \theta) } = \ans{\mathbb{E}_\rc{q} \ln \frac{p(\z\mid \x,  \theta)p(\x\mid \theta) }{ p(\z \mid \x, \theta)}}{\ldots}\\
&= \mathbb{E}_\rc{q} \ln p(\x \mid \theta)  = \ln p(\x \mid \theta)  
\end{align*}

In the EM algorithm, we search by alternately optimizing $L(\rc{q}\mid \theta)$ with respect to $\theta$ and setting $\rc{q}$ equal to $\gc{p}$ (so that the KL term becomes zero).

\qu For the variational autoencoder, we cannot (easily) perform this last step. Why not?
\ans{
In the variational autoencoder, our model is a neural network that transforms $\z$ into a distribution on $\x$. To set the KL divergence term equal to zero, we would have to compute $p(\z \mid \x, \theta)$: i.e. a probability distribution on $\z$ that indicates for which $\z$ our observed $\x$ is most likely.

While sampling techniques exist to approximate this kind of distribution, they are costly and can be very inaccurate.
}{}

Instead, we \emph{approximate} $\gc{p}(\z \mid\x, \theta)$ with a neural network $\rc{q}(\z \mid \x)$ that produces a distribution on $\z$ given some $\x$. This gives us an auto-encoder-like structure. An input is mapped to a distribution $\rc{q}(\z\mid \x)$ by the \emph{encoder}. We sample a single $\z$ from this distribution and pass it through the \emph{decoder} $p(\x \mid \z, \theta)$ to produce a distribution on $\x$ (see the \href{https://www.dropbox.com/s/bsjiwug8ykk33t6/51.DeepLearning2.annotated.pdf?dl=0}{slides} for diagrams).

To find a way to train such an architecture, we turn again to our decomposition of the likelihood:
\[
\ln p(\x\mid \theta) = L(\rc{q}, \theta) + KL(\rc{q}, \gc{p}) \p \\
\]
Since the KL divergence is always positive, this tells us that 

\[
\ln p(\x\mid \theta) \geq L(\rc{q}, \theta) \\
\]
for \emph{any} $\rc{q}$ we choose. This is why $L$ is called the variational \emph{lower bound}.\footnotemark If we choose our parameters $\theta$ and our function $\rc{q}$ to maximize $L$, we are also, indirectly, maximizing $p(\x\mid \theta)$.\footnotemark

\footnotetext{The word \emph{variational} comes from the fact that one of its arguments, \rc{q}, is a function (the calculus of functions is called \emph{variational} calculus). For our purposes, this distinction doesn't matter much, since the function $\rc{q}$ is defined by a set of parameters, so ultimately we will take the derivative over those parameters, as we are used to.}

\footnotetext{How close the lower bound $L$ comes to the true value $p(\x\mid \theta)$ depends on how well our encoder network $\rc{q}$ approximates the true conditional distribution on $\z$: $\gc{p}(\z \mid\x, \theta)$.}

To do so, we rewrite it into two separate terms: a KL divergence and an expectation:

\[
L(\rc{q}, \theta) = - KL(\rc{q}(\z\mid\x), p(\z)) + \E_\rc{q} \ln p(\x\mid \z, \theta)
\]

\qu Show that this equation holds (i.e. rewrite the left part into the right).
\ans{
We will omit the dependency of $p$ on $\theta$, to clarify the notation.
\begin{align*}
	L(\rc{q}, \theta) &= \E_\rc{q} \ln \frac{p(\x, \z)}{\rc{q}(\z\mid \x)} \\
	&= \E \ln p(\x, \z) - \E \rc{q}(\z \mid \x) \\
	&= \E \ln \left[p(\x\mid \z)p(\z)\right] - \E \ln \rc{q}(\z \mid \x) \\
	&= \E \ln p(\x \mid \z) + \E \ln p(\z) - \E \ln \rc{q}(\z \mid \x) \\
	&= \E \ln p(\x \mid \z) - \left [ \E \ln \rc{q}(\z \mid \x) - \E \ln p(\z) \right ]\\
	&= - KL(\rc{q}(\z\mid\x), p(\z)) + \E \ln p (\x\mid \z)
\end{align*}
}{}

Thus, to optimize our variational autoencoder, we should maximize $L$. In other words, $-L$ is our loss function. The only problem left to solve is that the second term is an expectation (which we cannot compute explicitly). 
\qu How is this solved in practice?

\ans{We take a single sample from $\rc{q}(\z\mid \x)$ and use $\ln p(\x\mid \x)$ as a (very crude) estimate of the expectation term. To let the gradient propagate through the sampling, we add a sample from the standard MVN to the input and transform it to a sample from $\rc{q}$ by multiplying by a matrix $A$ (with $\Sigma = AA^T$) and adding the mean.}{}

\end{document}

























